\usepackage[english]{babel}
\usepackage[utf8]{inputenc}
\usepackage[singlelinecheck=false]{caption}
\usepackage{lipsum}
\usepackage{listings}
\usepackage{shortvrb}
\usepackage{stfloats}
\usepackage{hyperref}
\usepackage{multicol}
\usepackage[detect-all]{siunitx}

\hypersetup{
    pdfborderstyle={/S/U/W 1}, % underline links instead of boxes
    linkbordercolor=red,       % color of internal links
    citebordercolor=green,     % color of links to bibliography
    filebordercolor=magenta,   % color of file links
    urlbordercolor=brown        % color of external links
}

\captionsetup[table]{labelformat=empty,font={sf,sc,bf,},skip=0pt}
\MakeShortVerb{|}

\lstset{%
  basicstyle=\ttfamily,
  language=[LaTeX]{TeX},
  breaklines=true,
}

\sisetup{range-phrase = \text{ to }}

\newcommand{\version}{v5.0}
\date{September 2022\newline{}\version}
\author{Daniel}

\title{
\Huge{Ginger Tea Solo Variations}
\\ \small{A \emph{Mythic}, \emph{Morning Coffee Solo Variations}, \& \emph{Plot Unfolding Machine} Mashup}
}

%----------------------------------------------------------------------------------
% Front
%----------------------------------------------------------------------------------
\begin{document}
\frontmatter
% \maketitle
% \tableofcontents

%----------------------------------------------------------------------------------
% Main Content
%----------------------------------------------------------------------------------
\mainmatter

\chapter*{Ginger Tea Solo Variations \version}
%----------------------------------------------------------------------------------
% Section: Introduction
%----------------------------------------------------------------------------------
\section{Introduction}
\DndDropCapLine{I}{have used the
\href{https://www.wordmillgames.com/mythic-gme.html}{Mythic}}
system for a long time. I love the creativity and modularity of that system,
but I have always felt just a little unsatisfied with the way that the chaos
factor changes the likelihood of getting a yes. I also wanted a system
that was simpler to use, with less reliance on modifiers, multiple dice, and
tracking of numeric chaos factors.

\href{https://aleaiactandaest.blogspot.com/p/downloads.html}{Morning Coffee Solo
Variations} (MCSV) comes very close to what I wanted. Rather than a moving chaos
factor and modifiers, it has an elegant system of chaos dice. When the chaos
factor changes, move the chaos die up or down a step ranging from d4 to d20. The
chaotic outcomes for scene setup and Yes/No questions use fixed numbers for each
result. This makes the outcomes easier to remember. However, MCSV just gives the
\emph{and ...}, \emph{but ...}, and \emph{random event} modifiers for Yes/No
questions. I always liked the \emph{exceptional} results in Mythic, so I have
modified the qualification table to add \emph{exceptional} as an option.

Additionally, the probabilities of \emph{and ...} and \emph{but ...}
modifiers in MCSV are much higher than I like. In this system they occur half
as often, but combined with the \emph{exceptional} result, there are still
plenty of things happening.

I also like some ideas in the excellent
\href{https://jeansenvaars.itch.io/plot-unfolding-machine}{Plot~Unfolding~
Machine} (PUM), so I have incorporated a few of those here in optional tables.
You need your own copy of PUM to use these, as my tables direct you to look
up additional results in PUM. These are all optional, so this system
can work without PUM as well. Just don't use the optional scene alteration
tables.

In the spirit of Morning Coffee Solo Variations, I have called this mashup the
\emph{Ginger Tea Solo Variations} (GTSV), since I was drinking ginger tea at
least some of the time while working on this.

%----------------------------------------------------------------------------------
\subsection{Chaos Factor}
The Chaos Factor table and mechanic is from
\href{https://aleaiactandaest.blogspot.com/p/downloads.html}{Morning Coffee Solo
Variations}. Generally speaking, high chaos (here represented by a smaller die) means things
are going badly for your PC. Low chaos means things are going well.

In this system, the chaos factor influences the likelihood of scenes running as
planned, as well as the likelihood of modifiers and random events on Yes/No
questions. Unlike Mythic, it does not change the likelihood of getting a yes or
a no.

If you are just starting an adventure and don't know what the chaos level is,
then \emph{Average (d10)} is a good choice.

\begin{DndTable}{X X}
    \textbf{Chaos Factor} & \textbf{Chaos Die} \\
    Boring$^a$ & d20\\
    Under Control & d12\\
    Average & d10\\
    Out of Control & d8\\
    Madness & d6\\
    Abject Chaos$^a$ & d5$^b$\\
    Plaything of the Gods$^a$ & d4\\
\end{DndTable}
\begin{scriptsize}
\-\vspace{-3mm}\linebreak
\-\hspace{0mm}$a$ Optional chaos settings\linebreak
\-\hspace{0mm}$b$ Roll a d10 and divide by 2, rounding up. Or just skip the d5.\par
\end{scriptsize}


\subsection{Scene Setup}
After setting up your scene, roll the chaos die against this table to test your
expectations. This table generates interrupt, altered, and unmodified scenes
with comparable frequencies to the original
\href{https://www.wordmillgames.com/mythic-gme.html}{Mythic system}.
\begin{DndTable}[header=Table 1: Classic]{c X}
    \textbf{Chaos} & \textbf{Outcome} \\
    $1, 2$ & Interrupt\\
    $3, 4$ & Altered\\
    $5+$ & As expected
\end{DndTable}

\section{Oracles}
\subsection{Yes or No}
The Yes/No oracle is used to answer questions about your RPG world, the
characters, and events within it. Unlike Mythic, the odds of yes or no are not
altered by the chaos factor.

First determine the odds of success, then roll a d6 and the chaos die. In the
event that the chaos die is a d6, use different colors to tell the dice apart.

\subsubsection{Standard Oracle}
\begin{DndTable}[header=Outcome (1d6)]{X X}
    \textbf{Odds} & \textbf{Yes if Oracle die rolls} \\
    Has to be & $2+^a$\\
    Very likely & $2+$\\
    Likely & $3+$\\
    Unsure & $4+$\\
    Unlikely & $5+$\\
    Very unlikely & $6$\\
    Impossible & $6^b$
\end{DndTable}
\begin{scriptsize}
\-\vspace{-3mm}\linebreak
\-\hspace{0mm}$a$ Roll 2d6 and discard the \textbf{\emph{lowest}} die before checking for complications.\linebreak
\-\hspace{0mm}$b$ Roll 2d6 and discard the \textbf{\emph{highest}} die before checking for complications.\par
\end{scriptsize}

\begin{DndTable}[header=Complications (Chaos Die)]{X X}
    \textbf{Chaos die} & \textbf{Complication} \\
    $1$ & And something good\ldots\\
    $2$ & And something bad\ldots\\
    $3+$ & No complication\\
    Oracle and Chaos die match & Random event
\end{DndTable}

\section{Optional Tables}
\subsection{Scene Setup}

\begin{DndComment}{}
You will need a copy of the
\href{https://jeansenvaars.itch.io/plot-unfolding-machine}{Plot~Unfolding~
Machine} to use these optional scene setup tables.
\end{DndComment}

\subsubsection{Classic with a Twist}
This scene setup table adds complications and challenges from the
\href{https://jeansenvaars.itch.io/plot-unfolding-machine}{Plot Unfolding
Machine} v2 to the Mythic altered and interrupt scenes.

A classic feel with a new twist.
\begin{DndTable}{c X}
    \textbf{Chaos} & \textbf{Outcome} \\
    $1$ & Unexpected complication$^a$\\
    $2$ & Interrupt\\
    $3$ & Altered\\
    $4$ & More challenging$^b$\\
    $\numrange{5}{11}$ & As expected\\
    $12+$ & Even better$^c$\\
\end{DndTable}
\begin{scriptsize}
\-\vspace{-4mm}\linebreak
\-\hspace{0mm}$^a$ Roll on PUM \emph{Scene Complication table}, consult Mythic
detail tables, or otherwise add complications.\linebreak
\-\hspace{0mm}$^b$ Roll on PUM \emph{Challenge Type} \& \emph{High Stakes}
tables, add a skill challenge, or somehow make the scene more challenging.\linebreak
\-\hspace{0mm}$^c$Similar to the \emph{PC Positive} events in Mythic.\par
\end{scriptsize}

\subsubsection{Narrative}
This table generates outcomes from Plot Unfolding Machine v3. This is the least
Mythic-like option. The options here have a narrative feel, putting plot,
action, characters, and conflict at the forefront.
\begin{DndTable}{c X}
    \textbf{Chaos} & \textbf{Outcome} \\
    $1$ & \textbf{\emph{Subject}} is \textbf{\emph{Revelation}}\\
    $2$ & Consider \textbf{\emph{Circumstance}}\\
    $3$ & The area is \textbf{\emph{Describe}}\\
    $4$ & \textbf{\emph{Who}} shows up, and \textbf{\emph{Intent}}\\
    $\numrange{5}{11}$ & As expected\\
    $12+$ & And also \textbf{\emph{Goal}}
\end{DndTable}
\begin{scriptsize}
\-\vspace{-5mm}\linebreak
\-\hspace{2mm}\textbf{\emph{Bold and italicized}} items indicate the tables to roll in PUM v3.
\end{scriptsize}

\appendix

%----------------------------------------------------------------------------------
% Comparison of probabilities
%----------------------------------------------------------------------------------
\onecolumn
\chapter{Probabilities}
\section{Yes/No Oracle}
\subsection{Chance of a Yes}
These tables show that within the range of \emph{doubtful} to \emph{certain}
(Mythic \emph{very unlikely} to \emph{likely}), the MCSV and GTSV Yes/No oracles
have roughly chances of a \emph{Yes} result as the Mythic Variations 2 Fate Check.

\begin{multicols}{2}
\clearfloat
\vspace*{\fill}
% Mythic
\begin{DndTable}[header=\emph{Mythic Variations 2 Fate Check}\\at Chaos Factor 5]{X c}
    \textbf{Odds} & \textbf{Chance of Yes} \\
    Impossible & $3\%$ \\
    No way & $10\%$ \\
    Very unlikely & $21\%$\\
    Unlikely & $36\%$\\
    Unsure & $55\%$\\
    Likely & $72\%$\\
    Very likely & $85\%$\\
    Sure thing & $94\%$\\
    Has to be & $99\%$
\end{DndTable}

% GTSV
\begin{DndTable}[header=GTSV]{X c}
    \textbf{Odds} & \textbf{Chance of Yes} \\
    Doubtful & $17\%$ \\
    Unlikely & $33\%$ \\
    Unsure & $50\%$ \\
    Likely & $67\%$ \\
    Certain & $83\%$
\end{DndTable}

% Closest Named Probabilities
\begin{DndTable}[header=Closest Equivalent Named Probabilities]{X l}
    \textbf{Mythic Fate Check} & \textbf{GTSV} \\
    Very unlikely ($21\%$)& Doubtful ($17\%$)\\
    Unlikely ($36\%$) & Unlikely ($33\%$)\\
    Unsure ($55\%$) & Unsure ($50\%$)\\
    Likely ($72\%$) & Likely ($67\%$)\\
    Very likely ($85\%$) & Certain ($83\%$)\\
\end{DndTable}
\end{multicols}

\subsection{Chance of a Modifier}
In all tables the highest chaos factor is on the left, decreasing towards the
lowest setting on the right.

As with the chance of a yes, the chance of answer modification within the range of
\emph{doubtful} to \emph{certain} (Mythic \emph{very unlikely} to \emph{likely})
is very similar between Mythic and GTSV.

\begin{DndComment}{A Note on Column Totals}
    In these tables, adding all percentages in a column should total to 100\%
    since this is the total chance of anything happening at all. In some cases
    it may appear that the total is not 100\%. There are a few reasons for this:
    \begin{itemize}
        \item I have rounded the fractional results to whole numbers. Rounding
        errors then make it appear as though things don't quite add up. The
        software I use to calculate these results gives the precise fractional
        odds. For example there is a $1/48$ chance of a \emph{random \&
        exceptional} outcome with chaos die d8.
        \item In MCSV and GTSV, random events occur independently of other outcomes.
        The chance of the four main outcomes totals $100\%$ while the random events
        have their own separate pool. They occur or not in combination with one of
        the four main outcomes.
        \item The \emph{Total Random Event Chance} in the GTSV table is the
        total chance of any random event happening at all. This is the sum of
        all the separate random event combinations.
    \end{itemize}
\end{DndComment}


\begin{DndTable}[header=\emph{Mythic Variations 2 Fate Check}]{l c c c c}
    \textbf{Modifier} & \textbf{Chaos 6} & \textbf{Chaos 5} & \textbf{Chaos 4} & \textbf{Chaos 3}\\
    Unmodified              & $ 64\%$          & $ 70\%$             & $ 76\%$               & $ 82\%$  \\
    Exceptional             & $15\%$           & $ 12\%$             & $ 10\%$               & $  8\%$  \\
    Random \& Unmodified    & $15\%$           & $ 12\%$             & $ 10\%$               & $  8\%$  \\
    Exceptional \& Random   & $ 6\%$           & $ 5\%$              & $ 4\%$                & $ 3\%$  \\
\end{DndTable}

\begin{DndTable}[header=GTSV Standard]{l c c c c c c c}
    \textbf{Modifier} & \textbf{d4} & \textbf{d5} & \textbf{d6} & \textbf{d8} & \textbf{d10} & \textbf{d12} & \textbf{d20}\\
    Unmodified              & $25\%$      & $40\%$       & $50\%$       & $62\%$       & $70\%$        & $75\%$        & $85\%$\\
    Exceptional             & $25\%$      & $20\%$       & $17\%$       & $12\%$       & $10\%$        & $ 8\%$        & $ 5\%$\\
    And ...                 & $25\%$      & $20\%$       & $17\%$       & $12\%$       & $10\%$        & $8\%$         & $5\%$\\
    But ...                 & $25\%$      & $20\%$       & $17\%$       & $12\%$       & $10\%$        & $8\%$         & $5\%$\\
    Total Random Event Chance& $17\%$      & $17\%$       & $17\%$       & $12\%$       & $10\%$        & $8\%$         & $5\%$\\
    Random \& Exceptional   & $ 4\%$      & $ 3\%$       & $ 3\%$       & $ 2\%$       & $ 2\%$        & $1\%$         & $1\%$\\
    Random \&~And...        & $ 4\%$      & $ 3\%$       & $ 3\%$       & $ 2\%$       & $ 2\%$        & $1\%$         & $1\%$\\
    Random \&~But...        & $ 4\%$      & $ 3\%$       & $ 3\%$       & $ 2\%$       & $ 2\%$        & $1\%$         & $1\%$\\
    Random \&~Unmodified    & $ 4\%$      & $ 7\%$       & $ 8\%$       & $ 6\%$       & $ 5\%$        & $4\%$         & $2\%$\\
\end{DndTable}

\section{Scene Setup}
\subsection{GTSV Scene Outcomes}

Note that I am primarily comparing to the newer \emph{Mythic Variations II Fate
Check}. In this system, the chaos factor only varies between 3 and 6.

Once again, within the range of \emph{doubtful} to \emph{certain} (Mythic
\emph{very unlikely} to \emph{likely}), the chances of scene alterations are
roughly similar when comparing Mythic and GTSV. They are not as close as with
the Yes/No oracle, but they are close enough to feel similar in play.

% Mythic odds
\begin{DndTable}[header=Mythic GME Scene Setup Probabilities]{l c c c c}
    \textbf{Modifier} & \textbf{Chaos 6} & \textbf{Chaos 5} & \textbf{Chaos 4} & \textbf{Chaos 3}\\
    Interrupt             & $30\%$           & $ 20\%$             & $ 20\%$               & $ 10\%$  \\
    Altered               & $30\%$           & $ 30\%$             & $ 20\%$               & $ 20\%$  \\
    Unmodified            & $40\%$           & $ 50\%$             & $ 60\%$               & $ 70\%$  \\
\end{DndTable}

% Classic table
\begin{DndTable}[header=Classic]{l c c c c c c c}
    \textbf{Outcome} & \textbf{d4} & \textbf{d5} & \textbf{d6} & \textbf{d8} & \textbf{d10} & \textbf{d12} & \textbf{d20}\\
    Interrupt        & $50\%$      & $40\%$        & $33\%$      & $25\%$        & $20\%$         & $17\%$         & $10\%$\\
    Altered          & $50\%$      & $40\%$        & $33\%$      & $25\%$        & $20\%$         & $17\%$         & $10\%$\\
    Unmodified       & $0\%$       & $20\%$        & $33\%$      & $50\%$        & $60\%$         & $67\%$         & $80\%$\\
\end{DndTable}

The following two tables can't really be compared to the previous results, since
the number of possible scene variation outcomes is much larger. However, the
general trend of reduced complications and increased good outcomes is seen as
chaos moves from high to low.

% Mythic & PUM v2
\begin{DndTable}[header=Classic with a Twist]{l c c c c c c c}
    \textbf{Outcome}        & \textbf{d4} & \textbf{d5} & \textbf{d6} & \textbf{d8} & \textbf{d10} & \textbf{d12} & \textbf{d20}\\
    Unexpected
    complication            & $25\%$        & $20\%$        & $17\%$        & $12\%$      & $10\%$         & $8\%$         & $5\%$\\
    Interrupt               & $25\%$        & $20\%$        & $17\%$        & $12\%$      & $10\%$         & $8\%$         & $5\%$\\
    Altered                 & $25\%$        & $20\%$        & $17\%$        & $12\%$      & $10\%$         & $8\%$         & $5\%$\\
    More
    challenging             & $25\%$        & $20\%$        & $17\%$        & $12\%$      & $10\%$         & $8\%$         & $5\%$\\
    As expected             & $0\%$         & $20\%$        & $33\%$        & $50\%$      & $60\%$         & $58\%$        & $35\%$\\
    Even better             & $0\%$         & $0\%$         & $0\%$         & $0\%$       & $0\%$          & $8\%$         & $45\%$\\
\end{DndTable}

% PUM v3
\begin{DndTable}[header=Narrative]{l c c c c c c c}
    \textbf{Outcome}                            & \textbf{d4} & \textbf{d5} & \textbf{d6} & \textbf{d8} & \textbf{d10} & \textbf{d12} & \textbf{d20}\\
    \textbf{Subject} is \textbf{Revelation}     & $25\%$        & $20\%$        & $17\%$        & $12\%$      & $10\%$         & $8\%$         & $5\%$\\
    Consider \textbf{Circumstance}              & $25\%$        & $20\%$        & $17\%$        & $12\%$      & $10\%$         & $8\%$         & $5\%$\\
    The area is \textbf{Describe}               & $25\%$        & $20\%$        & $17\%$        & $12\%$      & $10\%$         & $8\%$         & $5\%$\\
    \textbf{Who} shows up, \& \textbf{Intent}   & $25\%$        & $20\%$        & $17\%$        & $12\%$      & $10\%$         & $8\%$         & $5\%$\\
    As expected                                 & $0\%$         & $20\%$        & $33\%$        & $50\%$      & $60\%$         & $58\%$        & $35\%$\\
    And also \textbf{Goal}                      & $0\%$         & $0\%$         & $0\%$         & $0\%$       & $0\%$          & $8\%$         & $45\%$\\
\end{DndTable}

%----------------------------------------------------------------------------------
% Back
%----------------------------------------------------------------------------------
\backmatter

\end{document}
