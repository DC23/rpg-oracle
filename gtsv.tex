\usepackage[english]{babel}
\usepackage[utf8]{inputenc}
\usepackage[singlelinecheck=false]{caption}
\usepackage{lipsum}
\usepackage{listings}
\usepackage{shortvrb}
\usepackage{stfloats}
\usepackage{hyperref}
\usepackage{multicol}
\usepackage[detect-all]{siunitx}

\hypersetup{
    pdfborderstyle={/S/U/W 1}, % underline links instead of boxes
    linkbordercolor=red,       % color of internal links
    citebordercolor=green,     % color of links to bibliography
    filebordercolor=magenta,   % color of file links
    urlbordercolor=brown        % color of external links
}

\captionsetup[table]{labelformat=empty,font={sf,sc,bf,},skip=0pt}
\MakeShortVerb{|}

\lstset{%
  basicstyle=\ttfamily,
  language=[LaTeX]{TeX},
  breaklines=true,
}

\sisetup{range-phrase = \text{ to }}

\newcommand{\version}{v5.0}
\date{September 2022\newline{}\version}
\author{Daniel}

\title{
\Huge{Ginger Tea Solo Variations}
\\ \small{A \emph{Mythic}, \emph{Morning Coffee Solo Variations}, \& \emph{Plot Unfolding Machine} Mashup}
}

%----------------------------------------------------------------------------------
% Front
%----------------------------------------------------------------------------------
\begin{document}
\onecolumn
\frontmatter{}

%----------------------------------------------------------------------------------
% Main Content
%----------------------------------------------------------------------------------
\mainmatter{}

\chapter*{Ginger Tea Solo Variations \version}
%----------------------------------------------------------------------------------
% Section: Introduction
%----------------------------------------------------------------------------------

\section{Introduction}
\DndDropCapLine{I}{have used the
\href{https://www.wordmillgames.com/mythic-gme.html}{Mythic}}
system for a long time. I love the creativity and modularity of that system,
but I have always felt just a little unsatisfied with the way that the chaos
factor changes the likelihood of getting a yes. I also wanted a system
that was simpler to use, with less reliance on modifiers, multiple dice, and
tracking of numeric chaos factors.

\href{https://aleaiactandaest.blogspot.com/p/downloads.html}{Morning Coffee Solo
Variations} (MCSV) comes very close to what I wanted. Rather than a moving chaos
factor and modifiers, it has an elegant system of chaos dice. When the chaos
factor changes, move the chaos die up or down a step ranging from d4 to d20. The
chaotic outcomes for scene setup and Yes/No questions use fixed numbers for each
result. This makes the outcomes easier to remember.

In the original MCSV, modifiers or complications occur too frequently for my
preference. I have reduced the odds to be a close match to Mythic.

In the spirit of Morning Coffee Solo Variations, I have called this mashup the
\emph{Ginger Tea Solo Variations} (GTSV), since I was drinking ginger tea at
least some of the time while working on this.

%----------------------------------------------------------------------------------
\begin{multicols}{2}
\clearfloat{}

\subsection{Chaos Factor}
Generally speaking, high chaos (here represented by a smaller die) means things
are going badly for your PC\@. Low chaos means things are going well.

In this system, the chaos factor influences the likelihood of scenes running as
planned, as well as the likelihood of modifiers and random events on Yes/No
questions. Unlike Mythic, it does not change the likelihood of getting a yes or
a no when asking an oracle question.

If you are just starting an adventure and don't know what the chaos level is,
then \emph{Average (d10)} is a good choice.

\begin{DndTable}{X X}
    \textbf{Chaos Factor} & \textbf{Chaos Die} \\
    Boring$^a$ & d20\\
    Under Control & d12\\
    Average & d10\\
    Out of Control & d8\\
    Madness & d6\\
    Abject Chaos$^a$ & d5$^b$\\
    Plaything of the Gods$^a$ & d4\\
\end{DndTable}
\begin{scriptsize}
\-\vspace{-3mm}\linebreak
\-\hspace{0mm}$a$ Optional chaos settings\linebreak
\-\hspace{0mm}$b$ Roll a d10 and divide by 2, rounding up. Or just skip the d5.\par
\end{scriptsize}


\subsection{Scene Setup}
After setting up your scene, roll the chaos die against this table to test your
expectations. This table generates interrupt, altered, and unmodified scenes
with comparable frequencies to the original
\href{https://www.wordmillgames.com/mythic-gme.html}{Mythic system}.
\begin{DndTable}{c X}
    \textbf{Chaos} & \textbf{Outcome} \\
    $\numrange{1}{2}$ & Interrupt\\
    $\numrange{3}{4}$ & Altered\\
    $5+$ & As expected
\end{DndTable}

\subsection{Oracle}
The Yes/No oracle is used to answer questions about your RPG world, the
characters, and events within it. Unlike Mythic, the odds of yes or no are not
altered by the chaos factor.

First determine the odds of success, then roll a d6 and the chaos die. In the
event that the chaos die is a d6, use different colors to tell the dice apart.

Note that the \emph{Extreme Yes} and \emph{Extreme No} results of Mythic are
hiding in this system. If the outcome is positive and you get \emph{and
something good\ldots}, then that's an \emph{Extreme Yes}. If the outcome is
negative and you get an \emph{and something bad\ldots}, then that's an
\emph{Extreme No}. Unlike Mythic, this system also gives mixed results that can
be interpreted as \emph{success with a cost} and \emph{failure with a
consolation}.

\begin{DndTable}[header=Outcome (1d6)]{X X}
    \textbf{Odds} & \textbf{Yes if Oracle die rolls} \\
    Has to be & $2+^a$\\
    Very likely & $2+$\\
    Likely & $3+$\\
    Unsure & $4+$\\
    Unlikely & $5+$\\
    Very unlikely & $6$\\
    Impossible & $6^b$
\end{DndTable}
\begin{scriptsize}
\-\vspace{-3mm}\linebreak
\-\hspace{0mm}$a$ Roll 2d6 and discard the \textbf{\emph{lowest}} die before checking for complications.\linebreak
\-\hspace{0mm}$b$ Roll 2d6 and discard the \textbf{\emph{highest}} die before checking for complications.\par
\end{scriptsize}

\begin{DndTable}[header=Complications (Chaos Die)]{X X}
    \textbf{Chaos die} & \textbf{Complication} \\
    $1$ & And something good\ldots\\
    $2$ & And something bad\ldots\\
    $3+$ & No complication\\
    Oracle and Chaos die match & Random event
\end{DndTable}

\vfill
\end{multicols}

\appendix

%----------------------------------------------------------------------------------
% Comparison of probabilities
%----------------------------------------------------------------------------------
\onecolumn
\chapter{Probabilities}
\section{Yes/No Oracle}
\subsection{Chance of a Yes}
This table shows that the GTSV Yes/No oracle has roughly the same chances of a
\emph{Yes} result as the Mythic Variations 2 Fate Check for the same named odds.

\begin{DndTable}[header=\emph{Chance of a Yes} at Chaos Factor 5]{XXX}
    \textbf{Odds} & \textbf{Mythic Variations Fate Check} & \textbf{GTSV} \\
    Impossible & $3\%$ & $3\%$ \\
    Nearly impossible & $10\%$ & --- \\
    Very unlikely & $21\%$ & $17\%$ \\
    Unlikely & $36\%$ & $33\%$ \\
    Unsure & $55\%$ & $50\%$ \\
    Likely & $72\%$ & $67\%$ \\
    Very likely & $85\%$ & $83\%$ \\
    Nearly certain & $94\%$ & --- \\
    Certain & $99\%$ & $97\%$
\end{DndTable}

% \begin{multicols}{2}
% \clearfloat{}

% \end{multicols}

\subsection{Chance of a Modifier}
In all tables the highest chaos factor is on the left, decreasing towards the
lowest setting on the right.

As with the chance of a yes, the chance of getting an exceptional or unmodified result
is very similar between Mythic and GTSV across all comparable chaos levels. The chance of a
random event is slightly lower in GTSV\@.

Note that when comparing the modifier probabilities, the approximate match between
\emph{Mythic Variations II} chaos factors and GTSV chaos is:
\begin{DndTable}[header=Equivalent Chaos Factors]{ XXX }
    \textbf{Chaos Level} & \textbf{Mythic Fate Check} & \textbf{GTSV} \\
    Under Control & 3 & d12 \\
    Average & 4 & d10 \\
    Out of Control & 5 & d8 \\
    Madness & 6 & d6
\end{DndTable}

\begin{DndComment}{A Note on Column Totals}
    In the following tables, adding all percentages in a column should total to 100\%
    since this is the total chance of anything happening at all. In some cases
    it may appear that the total is not 100\%. There are a few reasons for this:
    \begin{itemize}
        \item I have rounded the fractional results to whole numbers. Rounding
        errors then make it appear as though things don't quite add up. The
        software I use to calculate these results gives the precise fractional
        odds. For example there is a $1/48$ chance of a \emph{random \&
        exceptional} outcome with chaos die d8.
        \item In MCSV and GTSV, random events occur independently of other outcomes.
        The chance of the main outcomes totals $100\%$ while the random events
        have their own separate pool. They occur or not in combination with one of
        the main outcomes.
        \item The \emph{Random} chance in the GTSV table is the
        total chance of any random event happening at all. This is the sum of
        all the separate random event combinations.
    \end{itemize}
\end{DndComment}

\vfill
\pagebreak

\begin{DndTable}[header=\emph{Mythic Variations 2 Fate Check}]{l c c c c}
    \textbf{Modifier} & \textbf{Chaos 6} & \textbf{Chaos 5} & \textbf{Chaos 4} & \textbf{Chaos 3}\\
    Unmodified              & $ 64\%$          & $ 70\%$             & $ 76\%$               & $ 82\%$  \\
    Exceptional             & $15\%$           & $ 12\%$             & $ 10\%$               & $  8\%$  \\
    Random                  & $21\%$           & $ 18\%$             & $ 14\%$               & $ 12\%$
\end{DndTable}

\begin{DndTable}[header=GTSV]{ lccccccc }
    \textbf{Modifier} & \textbf{d4} & \textbf{d5} & \textbf{d6} & \textbf{d8} & \textbf{d10} & \textbf{d12} & \textbf{d20}\\
    Unmodified              & $50\%$      & $60\%$       & $67\%$       & $75\%$       & $80\%$        & $83\%$        & $90\%$\\
    Exceptional             & $25\%$      & $20\%$       & $17\%$       & $12\%$       & $10\%$        & $8\%$         & $5\%$\\
    But \ldots              & $25\%$      & $20\%$       & $17\%$       & $12\%$       & $10\%$        & $8\%$         & $5\%$\\
    Random                  & $17\%$      & $17\%$       & $17\%$       & $12\%$       & $10\%$        & $8\%$         & $5\%$
\end{DndTable}

\section{Scene Setup}
\subsection{GTSV Scene Outcomes}

Note that I am primarily comparing to the newer \emph{Mythic Variations II Fate
Check}. In this system, the chaos factor only varies between 3 and 6.

Once again, within the range of \emph{doubtful} to \emph{certain} (Mythic
\emph{very unlikely} to \emph{likely}), the chances of scene alterations are
roughly similar when comparing Mythic and GTSV\@. They are not as close as with
the Yes/No oracle, but they are close enough to feel similar in play.

% Mythic odds
\begin{DndTable}[header=Mythic GME Scene Setup Probabilities]{l c c c c}
    \textbf{Modifier} & \textbf{Chaos 6} & \textbf{Chaos 5} & \textbf{Chaos 4} & \textbf{Chaos 3}\\
    Interrupt             & $30\%$           & $ 20\%$             & $ 20\%$               & $ 10\%$  \\
    Altered               & $30\%$           & $ 30\%$             & $ 20\%$               & $ 20\%$  \\
    Unmodified            & $40\%$           & $ 50\%$             & $ 60\%$               & $ 70\%$  \\
\end{DndTable}

% Classic table
\begin{DndTable}[header=Classic]{l c c c c c c c}
    \textbf{Outcome} & \textbf{d4} & \textbf{d5} & \textbf{d6} & \textbf{d8} & \textbf{d10} & \textbf{d12} & \textbf{d20}\\
    Interrupt        & $50\%$      & $40\%$        & $33\%$      & $25\%$        & $20\%$         & $17\%$         & $10\%$\\
    Altered          & $50\%$      & $40\%$        & $33\%$      & $25\%$        & $20\%$         & $17\%$         & $10\%$\\
    Unmodified       & $0\%$       & $20\%$        & $33\%$      & $50\%$        & $60\%$         & $67\%$         & $80\%$\\
\end{DndTable}

%----------------------------------------------------------------------------------
% Back
%----------------------------------------------------------------------------------
\backmatter{}

\end{document}
