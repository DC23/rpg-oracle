\usepackage[english]{babel}
\usepackage[utf8]{inputenc}
\usepackage[singlelinecheck=false]{caption}
\usepackage{listings}
\usepackage{shortvrb}
\usepackage{stfloats}
\usepackage{hyperref}

\hypersetup{
    pdfborderstyle={/S/U/W 1}, % underline links instead of boxes
    linkbordercolor=red,       % color of internal links
    citebordercolor=green,     % color of links to bibliography
    filebordercolor=magenta,   % color of file links
    urlbordercolor=brown        % color of external links
}

\captionsetup[table]{labelformat=empty,font={sf,sc,bf,},skip=0pt}
\MakeShortVerb{|}

\lstset{%
  basicstyle=\ttfamily,
  language=[LaTeX]{TeX},
  breaklines=true,
}

% Metadata
\title{The Oracle of Daniel}
\date{September 2022\newline{}Version 0.1}
% \author{Daniel}

%----------------------------------------------------------------------------------
% Front
%----------------------------------------------------------------------------------
\begin{document}
% \frontmatter
% \maketitle
% \tableofcontents

%----------------------------------------------------------------------------------
% Main Content
%----------------------------------------------------------------------------------
\mainmatter
\section{Oracles}
\subsection{Chaos Factor}
The Chaos Factor table and mechanic is from
\href{https://aleaiactandaest.blogspot.com/p/downloads.html}{Morning Coffee Solo
Variations}.

Generally speaking, high chaos (here represented by a smaller die) means things
are going badly for your PC. Low chaos means things are going well.

In this system, the chaos factor influences the likelihood of scenes running as
planned, as well as the likelihood of modifiers and random events on Yes/No
questions.

\begin{DndTable}[header=Chaos Factors]{X X}
    \textbf{Chaos Factor} & \textbf{Chaos Die} \\
    Boring$^a$ & d20\\
    Under Control & d12\\
    Average & d10\\
    Out of Control & d8\\
    Madness & d6\\
    Abject Chaos$^a$ & d5$^b$\\
    Plaything of the Gods$^a$ & d4\\
\end{DndTable}
\begin{scriptsize}
\-\vspace{-3mm}\linebreak
\-\hspace{3mm}$^a$ Optional chaos settings\linebreak
\-\hspace{3mm}$^b$ Roll a d10 and divide by 2 rounding up\par
\end{scriptsize}

\subsection{Scene Setup}
After setting up your scene, roll the chaos die against one of these tables to
test your expectations. The first table only generates Interrupt and Altered scenes
with similar frequencies to the \href{https://www.wordmillgames.com/mythic-gme.html}{Mythic system}. The second table adds complications
and challenges from the \href{https://jeansenvaars.itch.io/plot-unfolding-machine}{Plot Unfolding Machine}.

\begin{DndTable}[header=Mythic-style Expectation Checker]{x X}
    \textbf{Chaos} & \textbf{Outcome} \\
    1, 2 & Something else happens (interrupt)\\
    3, 4 & Something is different (altered)\\
    5+ & As expected
\end{DndTable}
\begin{DndTable}[header=PUM v2 Expectation Checker]{x X}
    \textbf{Chaos} & \textbf{Outcome} \\
    1 & Unexpected complication! Roll on PUM \emph{Scene Complication table}\\
    2 & Something else happens (interrupt)\\
    3 & Something is different (altered)\\
    4 & Add a challenge. Roll on PUM \emph{Challenge Type} \& \emph{High Stakes} tables\\
    5+ & As expected\\
    10 & Even better
\end{DndTable}
\begin{DndTable}[header=PUM v3 Expectation Checker]{x X}
    \textbf{Chaos} & \textbf{Outcome} \\
    1 & \textbf{Subject} is \textbf{Revelation}\\
    2 & Consider \textbf{Circumstance}\\
    3 & The area is \textbf{Describe}\\
    4 & \textbf{Who} shows up, and \textbf{Intent}\\
    5+ & As expected\\
    10 & And also \textbf{Goal}
\end{DndTable}

\subsection{Yes or No}
Note that unlike Mythic, the chaos factor does not influence the odds
of getting a yes or no. It only changes the chance of getting a modifier.

First determine the odds of success, then roll a d6 and the chaos die.
In the event that the chaos die is a d6, use different colours to tell the dice apart.
\begin{DndTable}[header=Outcome (1d6)]{X X}
    \textbf{Odds} & \textbf{Yes if} \\
    Certain & 2+\\
    Likely & 3+\\
    50/50 & 4+\\
    Unlikely & 5+\\
    Doubtful & 6
\end{DndTable}

\begin{DndTable}[header=Qualifier]{X X}
    \textbf{Chaos Die} & \textbf{Qualification} \\
    1, 2 & Exceptional \\
    3 & And ... something good\\
    4 & But ... something bad\\
    5+ & Unmodified\\
    Oracle and Chaos die match & Random event
\end{DndTable}

\appendix

%----------------------------------------------------------------------------------
% Back
%----------------------------------------------------------------------------------
\backmatter

\end{document}
