\usepackage[english]{babel}
\usepackage[utf8]{inputenc}
\usepackage[singlelinecheck=false]{caption}
\usepackage{lipsum}
\usepackage{listings}
\usepackage{shortvrb}
\usepackage{stfloats}
\usepackage{hyperref}

\hypersetup{
    pdfborderstyle={/S/U/W 1}, % underline links instead of boxes
    linkbordercolor=red,       % color of internal links
    citebordercolor=green,     % color of links to bibliography
    filebordercolor=magenta,   % color of file links
    urlbordercolor=brown        % color of external links
}

\captionsetup[table]{labelformat=empty,font={sf,sc,bf,},skip=0pt}
\MakeShortVerb{|}

\lstset{%
  basicstyle=\ttfamily,
  language=[LaTeX]{TeX},
  breaklines=true,
}

% Metadata
\title{Mythic, PUM, \& MCSV Mashup}
\date{September 2022\newline{}Version 1.1}

%----------------------------------------------------------------------------------
% Front
%----------------------------------------------------------------------------------
\begin{document}
% \frontmatter
% \maketitle
% \tableofcontents

%----------------------------------------------------------------------------------
% Main Content
%----------------------------------------------------------------------------------
\mainmatter
\section{Introduction}
I have used the \href{https://www.wordmillgames.com/mythic-gme.html}{Mythic}
system for a long time. I love the creativity and modularity of that system,
but I have always felt just a little unsatisfied with the way that the chaos
factor changes the likelihood of getting a yes. Additionally, I wanted a system
that was simpler to use, with less reliance on modifiers, multiple dice, and
tracking of numeric chaos factors.

\href{https://aleaiactandaest.blogspot.com/p/downloads.html}{Morning Coffee Solo
Variations} (MCSV) comes very close to what I wanted. Rather than a chaos factor
that needs to be tracked, and that provides a moving target for interpretation
during rolls, it has an elegant system of chaos dice. When the chaos factor
changes, move the chaos die up or down a step from d4 to d20. The chaotic
outcomes for scene setup and Yes/No questions use fixed numbers for each result.
This makes the outcomes easier to remember. However, MCSV just gives the
\emph{and ...}, \emph{but ...}, and \emph{random event} modifiers for Yes/No
questions. I always liked the \emph{exceptional} results in Mythic, so I have
modified the qualification table to add \emph{exceptional} as an option.

So here I have mashed up some ideas from Mythic, MCSV, and the excellent
\href{https://jeansenvaars.itch.io/plot-unfolding-machine}{Plot Unfolding
Machine} (PUM), hopefully without infringing on the intellectual property of any
of those.

For scene setup, what PUM wonderfully calls \emph{Expectation Checking}, both
Mythic and MCSV stick with the options of \emph{interrupt scenes}, \emph{altered
scenes}, and \emph{as expected}. PUM adds a number of other interesting outcomes
which I have drawn on to add three different scene setup tables. You will need a
copy of PUM to use the PUM v3 table.

\begin{DndReadAloud}
I may add additional oracle tables as inspiration and whim strike, but I have
nothing planned right now.
\end{DndReadAloud}

\section{Oracles}
\subsection{Chaos Factor}
The Chaos Factor table and mechanic is from
\href{https://aleaiactandaest.blogspot.com/p/downloads.html}{Morning Coffee Solo
Variations}.

Generally speaking, high chaos (here represented by a smaller die) means things
are going badly for your PC. Low chaos means things are going well.

In this system, the chaos factor influences the likelihood of scenes running as
planned, as well as the likelihood of modifiers and random events on Yes/No
questions. Unlike Mythic, it does not change the likelihood of getting a yes or a no.

\begin{DndTable}[header=Chaos Factors]{X X}
    \textbf{Chaos Factor} & \textbf{Chaos Die} \\
    Boring$^a$ & d20\\
    Under Control & d12\\
    Average & d10\\
    Out of Control & d8\\
    Madness & d6\\
    Abject Chaos$^a$ & d5$^b$\\
    Plaything of the Gods$^a$ & d4\\
\end{DndTable}
\begin{scriptsize}
\-\vspace{-3mm}\linebreak
\-\hspace{0mm}$^a$ Optional chaos settings\linebreak
\-\hspace{0mm}$^b$ Roll a d10 and divide by 2, rounding up\par
\end{scriptsize}

\subsection{Scene Setup}
After setting up your scene, roll the chaos die against one of these tables to
test your expectations.
\begin{itemize}
\item \emph{Mythic} generates Interrupt and Altered scenes with similar
frequencies to the \href{https://www.wordmillgames.com/mythic-gme.html}{Mythic
system}.
\item \emph{Mythic \& PUM v2} adds complications and challenges from the
\href{https://jeansenvaars.itch.io/plot-unfolding-machine}{Plot Unfolding
Machine} v2 to the Mythic altered and interrupt scenes.
\item \emph{PUM v3} generates outcomes from Plot Unfolding Machine v3. This is
the least Mythic-like option.
\end{itemize}
\begin{DndComment}{}
You will need a copy of the
\href{https://jeansenvaars.itch.io/plot-unfolding-machine}{Plot Unfolding
Machine} to use the two PUM tables. This is particularly the case for the
\emph{PUM v3} table.
\end{DndComment}

\begin{DndTable}[header=Mythic]{x X}
    \textbf{Chaos} & \textbf{Outcome} \\
    1, 2 & Interrupt\\
    3, 4 & Altered\\
    5+ & As expected
\end{DndTable}
% \\ \newpage

\begin{DndTable}[header=Mythic \& PUM v2]{x X}
    \textbf{Chaos} & \textbf{Outcome} \\
    1 & Unexpected complication$^a$\\
    2 & Interrupt\\
    3 & Altered\\
    4 & More challenging$^b$\\
    5-11 & As expected\\
    12+ & Even better$^c$\\
\end{DndTable}
\begin{scriptsize}
\-\vspace{-4mm}\linebreak
\-\hspace{0mm}$^a$ Roll on PUM \emph{Scene Complication table}, consult Mythic
detail tables, or otherwise add complications.\linebreak
\-\hspace{0mm}$^b$ Roll on PUM \emph{Challenge Type} \& \emph{High Stakes}
tables, add a skill challenge, or somehow make the scene more challenging.\linebreak
\-\hspace{0mm}$^c$Similar to the \emph{PC Positive} events in Mythic.\par
\end{scriptsize}

\begin{DndTable}[header=PUM v3]{x X}
    \textbf{Chaos} & \textbf{Outcome} \\
    1 & \textbf{\emph{Subject}} is \textbf{\emph{Revelation}}\\
    2 & Consider \textbf{\emph{Circumstance}}\\
    3 & The area is \textbf{\emph{Describe}}\\
    4 & \textbf{\emph{Who}} shows up, and \textbf{\emph{Intent}}\\
    5+ & As expected\\
    12+ & And also \textbf{\emph{Goal}}
\end{DndTable}
\begin{scriptsize}
\-\vspace{-5mm}\linebreak
\-\hspace{2mm}\textbf{\emph{Bold and italicized}} items indicate the tables to roll in PUM v3.
\end{scriptsize}

\subsection{Yes or No}
First determine the odds of success, then roll a d6 and the chaos die.
In the event that the chaos die is a d6, use different colors to tell the dice apart.
\begin{DndTable}[header=Outcome (1d6)]{X X}
    \textbf{Odds} & \textbf{Yes if} \\
    Certain & 2+\\
    Likely & 3+\\
    50/50 & 4+\\
    Unlikely & 5+\\
    Doubtful & 6
\end{DndTable}

\begin{DndTable}[header=Qualifier]{X X}
    \textbf{Chaos Die} & \textbf{Qualification} \\
    1, 2 & Exceptional \\
    3 & And ... something good\\
    4 & But ... something bad\\
    5+ & Unmodified\\
    Oracle and Chaos die match & Random event
\end{DndTable}

\appendix

%----------------------------------------------------------------------------------
% Back
%----------------------------------------------------------------------------------
\backmatter

\end{document}
